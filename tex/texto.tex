\chapter{Introdução}

\section{Motivação}

As cidades são repletas de agentes, cada um com diferentes interesses. Os cidadãos desejam maximizar a qualidade de vida, o que geralmente significa habitar residências espaçosas, em lugares tranquilos, repletos de natureza, mas que ainda tenahm acesso a oportunidades de emprego, lazer, saúde, educação, entre outros. As imobiliárias e construtoras desejam construir prédios, casas e condomínios que maximizem seus lucros, escolhendo as localizações com maior demanda e menores custos. Os proprietários de terras e imóveis buscam comprar ativos baratos com a expectativa de que valorizem no futuro ou extraiam renda de aluguel. 

Porém, a dinâmica das cidades é moldada por inúmeros conflitos de interesses, visto que muitos deles são incompatíveis. É impossível, por exemplo, que todo mundo more em contato com a natureza, já que na medida em que cada família constrói sua casa, a natureza do vizinho diminui. Dessa forma, há duas maneiras de coexistir no espaço: através das relações de poder, nas quais, o agente mais poderoso tem seu interesse atendido (ou negociado) em detrimento de outro, ou um planejador central intermedia este conflito de forma a gerar o resultado mais favorável para todos.

A principal instituição responsável por intermediar os conflitos no que tange a esfera urbana de São Paulo, a cidade mais populosa das américas, é o Plano Diretor Estratégico (PDE). O PDE é uma lei municipal que deve ser revisado a cada 10 anos, com uma revisão intermediária depois de 5 anos\footnote{Estas datas historicamente não foram respeitadas à risca. O PDE de 2002 ficou em vigor 12 anos, até a vigência do plano de 2014 começar. A revisão do PDE de 2014 ocorreu em 2023. Isso geralmente decorre dos atrasos administrativos  e debates prolongados}. O PDE de 2014 trouxe mudanças significativas em relação ao seu predecessor, sendo reconhecido internacionalmente pela sua abordagem de combate às desigualdades e incentivo ao uso do transporte público.

Enquanto o plano de 2002 descentralizou grande parte das decisões sobre zoneamento nas subprefeituras, o PDE de 2014 seguiu uma abordagem moderna, em linha com a literatura de urbanismo de \textit{Transit Oriented Development} (TOD), em português, ``desenvolvimento orientado ao transporte''. Antigamente, cada subprefeitura teria protagonismo no zoneamento de sua região, enquanto no novo plano, a regulação foi definida a nível municipal e centralizado. O sistema antigo apresentava grande dificuldade em lidar com os conflitos de interesse e grupos locais tinham a capacidade de conter o adensamento em áreas desejáveis, prejudicando o desenvolvimento sustentável da cidade.

A literatura de TOD, que surgiu no final da década de 80, mas popularizou-se apenas depois dos anos 2000, defende a mobilidade como um dos principais pilares do desenvolvimento das cidades \cite{Ibraeva2020}. A proposta do TOD é aproximar as famílias às oportunidades, através do direcionamento do desenvolvimento urbano no entorno de infraestrutura de transporte coletivo, causando uma melhora na mobilidade e maior adensamento. Dessa forma, diminui-se a dependência do carro, que gera externalidades negativas, e incentiva-se o uso do transporte público de alta capacidade, cujas externalidades são mais positivas, ou menos negativas do que o transporte individual motorizado.

As ideias do TOD não apenas influenciam o PDE de 2014, como são uma parte estruturante. Entre os objetivos enunciados do plano diretor, destacam-se os quatro primeiros:

\begin{quote}
    Art. 7º A Política de Desenvolvimento Urbano 
    e o Plano Diretor Estratégico se orientam pelos 
    seguintes objetivos estratégicos:

    I - conter o processo de expansão horizontal da aglomeração urbana, contribuindo para preservar o cinturão verde metropolitano;

    II - acomodar o crescimento urbano nas áreas subutilizadas dotadas de infraestrutura e no entorno da rede de transporte coletivo de alta e média capacidade

    III - reduzir a necessidade de deslocamento, equilibrando a relação entre os locais de emprego e de moradia;

    IV - expandir as redes de transporte coletivo de alta e média capacidade e os modos não motorizados, racionalizando o uso de automóvel;
\end{quote}

Dito isso, a pergunta motivadora desta pesquisa é se 10 anos depois do início da vigência do plano diretor é possível identificar se ele está cumprindo com seus objetivos -- em especial, o objetivo de adensar as áreas na proximidade da infraestrutura de transporte público.

\section{O plano diretor}

Esta seção é dedicada a compreender quais ferramentas estão a disposição do plano diretor para que alcance seus objetivos. Todavia, o PDE apresenta diversos instrumentos que agem não apenas em prédios novos, como também atua na requalificação de lotes já construídos. O foco desta pesquisa é nos três principais instrumentos desenhados para a regulamentação de novas construções. Estes são o coeficiente de aproveitamento, a cota parte e o gabarito. Ou outros intrumentos, apesar de importantes, não têm como objetivo incentivar ou desencentivar a densidade populacional de cada área. 

O PDE dividiu o mapa da cidade em macroáreas, macrozonas e zonas especiais, cada uma com objetivos específicos e respectivas restrições para cada instrumento. Além disso, criou um regime especial para lotes que estejam próximos à infraestrutura de transporte público de alta capacidade, região nomeada de Eixos de Estruturação da Transformação Urbana (EETUs). 

\subsection*{Eixos de Estruturação da Transformação Urbana (EETUs)}

Os eixos são ativados pela proximidade ao transporte público de alta capacidade, no caso de São Paulo, isso pode se dar de duas maneiras. A primeira forma de ativação, é através das estações de trens, metrôs ou monotrilhos, que criam uma área de influência em um raio de 400m ao redor do ponto de acesso à estação. A segunda forma é através de corredores de ônibus municipais e intermunicipais, que geram uma área de influência de 150m de distância para cada lado da via ao longo do corredor.

Nas zonas de eixos, além haver um regime especial para os três instrumentos que serão apresentados, há outras medidas que almejam incentivar o uso do transporte coletivo. Para que as pessoas utilizem menos o carro, por exemplo, criou-se restrições para o número de vagas de estacionamento, que antes eram incentivadas no PDE de 2002.

\subsection*{Coeficiente de Aproveitamento (CA)}

O Coeficiente de Aproveitamento (CA) é um instrumento utilizado mundialmente\footnote{Em inglês, \textit{Floor Area Ratio} (FAR) ou \textit{Floor Area Ratio}} em planos diretores para regular a \textbf{densidade construtiva}. O CA determina quantas vezes a área do lote pode ser construída. A legislação paulistana prescreve no direito de propriedade de todos os lotes da cidade um coeficiente de aproveitamento básico igual a 1, ou seja, é permitido construir uma vez a área do terreno. Se um proprietário desejar construir 4 andares em seu lote, com uma CA de 1, pode ocupar apenas 1/4 da área do terreno. Na ótica da legislação, o potencial de construir metros quadrados a mais de uma vez a área do terreno não está incluso no direito de propriedade, e, portanto pertence à sociedade, não ao proprietário do lote.

Em cada macroárea da cidade há restrições diferentes para o CA mínimo e máximo -- o básico se mantém 1 em toda a cidade. Na Macrozona de Estruturação e Qualificação Urbana, o CA mínimo varia entre 0,3 e 0,5 a depender de qual macroárea se encontra, e o máximo é sempre 2. Na Macrozona de Proteção e Recuperação Ambiental, não há CA mínimo e o CA máximo varia entre 0,1 e 1, a depender do nível de proteção que foi designado à região. Nessas áreas de preservação outras regulações também podem atuar\footnote{Aplica-se a legislação estadual pertinente, especialmente as leis específicas das Bacias Billings e Guarapiranga}.

Quando a região é ativada transporte público de alta capacidade, há modificadores sobre o CA. Na Macrozona de Estruturação e Qualificação Urbana, caso haja região de eixo, o CA máximo aumenta para 4. Na Macrozona de Proteção e Recuperação Ambiental, contanto que a área esteja fora da região de proteção aos mananciais, o CA máximo torna-se 2. Dessa forma, o PDE permite o dobro de densidade construtiva no entorno da infraestrutura de transporte, demonstrando seu compromisso com o TOD.

\subsection*{Gabarito}

O gabarito também é um instrumento comum de se regular em planos diretores para controlar a \textbf{verticalização} na cidade. O gabarito, dentre os instrumentos discutidos, é o mais visível ao olho das pessoas que passam na rua ou da vista aérea, quando se chega de avião no aeorporto de Guarulhos ou Congonhas. Na Macrozona de Estruturação e Qualificação Urbana, o gabarito máximo é de 28m ou, contabilizando-se em pavimentos, o térreo mais 8 andares. Nas áreas de Macrozona de Proteção e Recuperação Ambiental, o gabarito máximo é de 15m, ou térreo mais 4 pavimentos.

Como uma forma de incentivo à produção imobiliária nos eixos, na Macrozona de Estruturação e Qualificação Urbana não há limite de gabarito para as construções, que passam a ser limitadas apenas pelo CA. Os eixos em Macrozona de Proteção e Recuperação Ambiental apresentam gabarito máximo de 28m.

Um detalhe importante de se pontuar é que há uma relação de identidade entre CA, verticalização e taxa de ocupação. A taxa de ocupação é o percentual da área do terreno que recebe edificação. Por exemplo, em um terreno com CA igual a 4, caso sejam construídos 8 andares, a taxa de ocupação precisa ser 50\%. Analogamente, se definir a taxa de ocupação em 100\%, por exemplo, o número de pavimentos deve ser 4. Mais detalhes sobre esta discussão, bem como um comparativo entre a densidade construtiva e a verticalização encontram-se no Apêndice \ref{appendix:verticalizacao}.

\subsection*{Cota Parte}

A cota parte, por sua vez, não é um instrumento comum de se observar em planos diretores e pode ser considerada uma medida relativamente experimental -- diferentemente do CA e gabarito que são instrumentos consolidados. Este instrumento é responsável por regulamentar a \textbf{densidade habitacional}. Sabendo-se a cota parte ($Q$) e a área do terreno ($A_t$), é possível identificar o número mínimo de unidades habitacionais ($N_{min}$) que o lote deve apresentar, seguindo a equação \ref{eq:cotaparte} a seguir. 

\begin{equation}
    N_{min} = \frac{\text{CA}_{\text{utilizado}}}{\text{CA}_{max}}\cdot \frac{A_t}{Q}
    \label{eq:cotaparte}
\end{equation}

Um lote, por exemplo, de 1000$m^2$ com cota parte de 20, caso utilize o CA máximo, precisa construir no mínimo 50 unidades. É interessante notar que o CA do lote não interfere no número mínimo de unidades habitacionais que devem ser produzidas, ao menos que o CA utilizado seja menor do que o máximo, e, neste caso, o número de unidades mínimo que devem ser providas é menor. Todavia, a cota parte não é um instrumento que funciona para a cidade toda. Na verdade, a única região em que a cota parte atua é a região de EETUs, nas quais a cota parte é de 20 no caso da Macrozona de Estruturação e Qualificação Urbana e de 40 em Macrozona de Proteção e Recuperação Ambiental.


% O primeiro fator importante de compreender sobre o PDE, é que adotou um ``coeficiente de aproveitamento básico 1 para todo o território municipal significa que o proprietário de um lote urbano tem inerente ao seu direito de propriedade a possibilidade de construir uma vez a área de seu terreno''. Para cada região da cidade, o coeficiente de aproveitamento foi modificado para permitir ou proibir construção.ss

\section{O problema}

\chapter{Dados}

\section{Densidade}
\section{Padrões Construtivos}
\section{Regulação}

\chapter{Metodologia e Resultados}

\section{Passo 1: }
\section{Passo 2: }
\section{Passo 3: }

\chapter{Conclusão}